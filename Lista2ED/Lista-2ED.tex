% Options for packages loaded elsewhere
\PassOptionsToPackage{unicode}{hyperref}
\PassOptionsToPackage{hyphens}{url}
%
\documentclass[
  12pt,
]{article}
\usepackage{amsmath,amssymb}
\usepackage{iftex}
\ifPDFTeX
  \usepackage[T1]{fontenc}
  \usepackage[utf8]{inputenc}
  \usepackage{textcomp} % provide euro and other symbols
\else % if luatex or xetex
  \usepackage{unicode-math} % this also loads fontspec
  \defaultfontfeatures{Scale=MatchLowercase}
  \defaultfontfeatures[\rmfamily]{Ligatures=TeX,Scale=1}
\fi
\usepackage{lmodern}
\ifPDFTeX\else
  % xetex/luatex font selection
\fi
% Use upquote if available, for straight quotes in verbatim environments
\IfFileExists{upquote.sty}{\usepackage{upquote}}{}
\IfFileExists{microtype.sty}{% use microtype if available
  \usepackage[]{microtype}
  \UseMicrotypeSet[protrusion]{basicmath} % disable protrusion for tt fonts
}{}
\makeatletter
\@ifundefined{KOMAClassName}{% if non-KOMA class
  \IfFileExists{parskip.sty}{%
    \usepackage{parskip}
  }{% else
    \setlength{\parindent}{0pt}
    \setlength{\parskip}{6pt plus 2pt minus 1pt}}
}{% if KOMA class
  \KOMAoptions{parskip=half}}
\makeatother
\usepackage{xcolor}
\usepackage[margin=1in]{geometry}
\usepackage{color}
\usepackage{fancyvrb}
\newcommand{\VerbBar}{|}
\newcommand{\VERB}{\Verb[commandchars=\\\{\}]}
\DefineVerbatimEnvironment{Highlighting}{Verbatim}{commandchars=\\\{\}}
% Add ',fontsize=\small' for more characters per line
\usepackage{framed}
\definecolor{shadecolor}{RGB}{248,248,248}
\newenvironment{Shaded}{\begin{snugshade}}{\end{snugshade}}
\newcommand{\AlertTok}[1]{\textcolor[rgb]{0.94,0.16,0.16}{#1}}
\newcommand{\AnnotationTok}[1]{\textcolor[rgb]{0.56,0.35,0.01}{\textbf{\textit{#1}}}}
\newcommand{\AttributeTok}[1]{\textcolor[rgb]{0.13,0.29,0.53}{#1}}
\newcommand{\BaseNTok}[1]{\textcolor[rgb]{0.00,0.00,0.81}{#1}}
\newcommand{\BuiltInTok}[1]{#1}
\newcommand{\CharTok}[1]{\textcolor[rgb]{0.31,0.60,0.02}{#1}}
\newcommand{\CommentTok}[1]{\textcolor[rgb]{0.56,0.35,0.01}{\textit{#1}}}
\newcommand{\CommentVarTok}[1]{\textcolor[rgb]{0.56,0.35,0.01}{\textbf{\textit{#1}}}}
\newcommand{\ConstantTok}[1]{\textcolor[rgb]{0.56,0.35,0.01}{#1}}
\newcommand{\ControlFlowTok}[1]{\textcolor[rgb]{0.13,0.29,0.53}{\textbf{#1}}}
\newcommand{\DataTypeTok}[1]{\textcolor[rgb]{0.13,0.29,0.53}{#1}}
\newcommand{\DecValTok}[1]{\textcolor[rgb]{0.00,0.00,0.81}{#1}}
\newcommand{\DocumentationTok}[1]{\textcolor[rgb]{0.56,0.35,0.01}{\textbf{\textit{#1}}}}
\newcommand{\ErrorTok}[1]{\textcolor[rgb]{0.64,0.00,0.00}{\textbf{#1}}}
\newcommand{\ExtensionTok}[1]{#1}
\newcommand{\FloatTok}[1]{\textcolor[rgb]{0.00,0.00,0.81}{#1}}
\newcommand{\FunctionTok}[1]{\textcolor[rgb]{0.13,0.29,0.53}{\textbf{#1}}}
\newcommand{\ImportTok}[1]{#1}
\newcommand{\InformationTok}[1]{\textcolor[rgb]{0.56,0.35,0.01}{\textbf{\textit{#1}}}}
\newcommand{\KeywordTok}[1]{\textcolor[rgb]{0.13,0.29,0.53}{\textbf{#1}}}
\newcommand{\NormalTok}[1]{#1}
\newcommand{\OperatorTok}[1]{\textcolor[rgb]{0.81,0.36,0.00}{\textbf{#1}}}
\newcommand{\OtherTok}[1]{\textcolor[rgb]{0.56,0.35,0.01}{#1}}
\newcommand{\PreprocessorTok}[1]{\textcolor[rgb]{0.56,0.35,0.01}{\textit{#1}}}
\newcommand{\RegionMarkerTok}[1]{#1}
\newcommand{\SpecialCharTok}[1]{\textcolor[rgb]{0.81,0.36,0.00}{\textbf{#1}}}
\newcommand{\SpecialStringTok}[1]{\textcolor[rgb]{0.31,0.60,0.02}{#1}}
\newcommand{\StringTok}[1]{\textcolor[rgb]{0.31,0.60,0.02}{#1}}
\newcommand{\VariableTok}[1]{\textcolor[rgb]{0.00,0.00,0.00}{#1}}
\newcommand{\VerbatimStringTok}[1]{\textcolor[rgb]{0.31,0.60,0.02}{#1}}
\newcommand{\WarningTok}[1]{\textcolor[rgb]{0.56,0.35,0.01}{\textbf{\textit{#1}}}}
\usepackage{graphicx}
\makeatletter
\def\maxwidth{\ifdim\Gin@nat@width>\linewidth\linewidth\else\Gin@nat@width\fi}
\def\maxheight{\ifdim\Gin@nat@height>\textheight\textheight\else\Gin@nat@height\fi}
\makeatother
% Scale images if necessary, so that they will not overflow the page
% margins by default, and it is still possible to overwrite the defaults
% using explicit options in \includegraphics[width, height, ...]{}
\setkeys{Gin}{width=\maxwidth,height=\maxheight,keepaspectratio}
% Set default figure placement to htbp
\makeatletter
\def\fps@figure{htbp}
\makeatother
\setlength{\emergencystretch}{3em} % prevent overfull lines
\providecommand{\tightlist}{%
  \setlength{\itemsep}{0pt}\setlength{\parskip}{0pt}}
\setcounter{secnumdepth}{5}
\usepackage[OT4]{polski}
\usepackage[utf8]{inputenc}
\usepackage{graphicx}
\usepackage{float}
\usepackage{booktabs}
\usepackage{longtable}
\usepackage{array}
\usepackage{multirow}
\usepackage{wrapfig}
\usepackage{float}
\usepackage{colortbl}
\usepackage{pdflscape}
\usepackage{tabu}
\usepackage{threeparttable}
\usepackage{threeparttablex}
\usepackage[normalem]{ulem}
\usepackage{makecell}
\usepackage{xcolor}
\ifLuaTeX
  \usepackage{selnolig}  % disable illegal ligatures
\fi
\usepackage{bookmark}
\IfFileExists{xurl.sty}{\usepackage{xurl}}{} % add URL line breaks if available
\urlstyle{same}
\hypersetup{
  pdftitle={Raport lista 2},
  pdfauthor={Dominik Kowalczyk i Matylda Mordal},
  hidelinks,
  pdfcreator={LaTeX via pandoc}}

\title{Raport lista 2}
\usepackage{etoolbox}
\makeatletter
\providecommand{\subtitle}[1]{% add subtitle to \maketitle
  \apptocmd{\@title}{\par {\large #1 \par}}{}{}
}
\makeatother
\subtitle{Eksploracja danych}
\author{Dominik Kowalczyk i Matylda Mordal}
\date{2025-04-25}

\begin{document}
\maketitle

{
\setcounter{tocdepth}{2}
\tableofcontents
}
\section{Zadanie 2 - Analiza składowych głównych (Principal Component
Analysis
(PCA))}\label{zadanie-2---analiza-skux142adowych-gux142uxf3wnych-principal-component-analysis-pca}

Dane pochodzą z pliku CSV i zawierają informacje o jakości życia w
różnych miastach świata, które posłużą do analizy danych (z pliku
``uaScoresDataFrame.csv'' dostępnego pod linkiem:
\url{https://www.kaggle.com/datasets/orhankaramancode/city-quality-of-life-dataset}.

\begin{Shaded}
\begin{Highlighting}[]
\CommentTok{\#Wczytanie zbioru danych}
\NormalTok{danePCA }\OtherTok{\textless{}{-}} \FunctionTok{read.csv}\NormalTok{(}\AttributeTok{file=}\StringTok{"uaScoresDataFrame.csv"}\NormalTok{, }\AttributeTok{stringsAsFactors =} \ConstantTok{TRUE}\NormalTok{)}
\end{Highlighting}
\end{Shaded}

Zweryfikujmy i spawdźmy z jakimi danymi mamy doczynenia

\begin{longtable}[t]{rlll}
\caption{\label{tab:analiza-klas}Opis danych PCA \label{tab:OpisDanych}}\\
\toprule
Indeks & Nazwa zmiennej & Typ zmiennej & Opis zmiennej\\
\midrule
1 & X & integer & Indeks porządkowy\\
2 & UA\_Name & factor & Nazwa obszaru miejskiego\\
3 & UA\_Country & factor & Kraj obszaru miejskiego\\
4 & UA\_Continent & factor & Kontynent obszaru miejskiego\\
5 & Housing & numeric & Wskaźnik  standardu zamieszkania\\
\addlinespace
6 & Cost.of.Living & numeric & Wskaźnik kosztów życia\\
7 & Startups & numeric & Wskaźnik liczby startupów\\
8 & Venture.Capital & numeric & Wskaźnik kapitału wysokiego ryzyka\\
9 & Travel.Connectivity & numeric & Wskaźnik łączności podróżniczej\\
10 & Commute & numeric & Wskaźnik dojazdów do pracy\\
\addlinespace
11 & Business.Freedom & numeric & Wskaźnik swobody działalności gospodarczej\\
12 & Safety & numeric & Wskaźnik bezpieczeństwa\\
13 & Healthcare & numeric & Wskaźnik opieki zdrowotnej\\
14 & Education & numeric & Wskaźnik edukacji\\
15 & Environmental.Quality & numeric & Wskaźnik jakości środowiska\\
\addlinespace
16 & Economy & numeric & Wskaźnik ekonomii\\
17 & Taxation & numeric & Wskaźnik opodatkowania\\
18 & Internet.Access & numeric & Wskaźnik dostępu do internetu\\
19 & Leisure...Culture & numeric & Wskaźnik rekreacji i kultury\\
20 & Tolerance & numeric & Wskaźnik tolerancji\\
\addlinespace
21 & Outdoors & numeric & Wskaźnik aktywności na świeżym powietrzu\\
\bottomrule
\end{longtable}

Chcąc zastosować metodę PCA, musimy wyodrębnić cechy ilościowe

\begin{Shaded}
\begin{Highlighting}[]
\CommentTok{\#Wyodębnienie zmiennych ilościowych}
\NormalTok{daneIlosciowe }\OtherTok{\textless{}{-}}\NormalTok{ danePCA }\SpecialCharTok{\%\textgreater{}\%}
  \FunctionTok{select}\NormalTok{(}\FunctionTok{where}\NormalTok{(is.numeric)) }\SpecialCharTok{\%\textgreater{}\%}
  \FunctionTok{select}\NormalTok{(}\SpecialCharTok{{-}}\NormalTok{X)  }
\CommentTok{\#usuwa kolumnę X, która reprezentowała indeksy, które są zbędne w analizie }
\end{Highlighting}
\end{Shaded}

Dodatkowo, aby móc efektownie zastosować metodę PCA musimy stwierdzić
czy potrzebna jest standaryzacja naszych danych ilościowych. W tym celu
wyliczymy ich wariację oraz porównamy je wizualnie.

\begin{longtable}[t]{lr}
\caption{\label{tab:wariancje}Wariancje cech ilościowych \label{tab:tabela2}}\\
\toprule
Cecha & Wariancja\\
\midrule
Housing & 5.265\\
Cost.of.Living & 5.988\\
Startups & 4.635\\
Venture.Capital & 6.520\\
Travel.Connectivity & 4.375\\
\addlinespace
Commute & 2.320\\
Business.Freedom & 4.450\\
Safety & 3.051\\
Healthcare & 2.196\\
Education & 4.897\\
\addlinespace
Environmental.Quality & 4.840\\
Economy & 2.302\\
Taxation & 2.855\\
Internet.Access & 3.505\\
Leisure...Culture & 4.027\\
\addlinespace
Tolerance & 2.974\\
Outdoors & 2.534\\
\bottomrule
\end{longtable}

\begin{center}\includegraphics{Lista-2ED_files/figure-latex/wariancje-1} \end{center}

Jak widać na (WYKESIE!) rozrzut (wariancja) między cechami ilościowymi
jest bardzo chaotyczny, zróżnicowany - niektóre cechy mają duży zakres,
a inne są mocno skupione wokół środka. Stąd, można stwierdzić, że należy
przeprowadzić standaryzację danych.

\begin{center}\includegraphics{Lista-2ED_files/figure-latex/standaryzacja danych-1} \end{center}

Mając już ustandaryzowane dane, możemy przejść do wyliczania i
porównania składowych głównych.

\begin{center}\includegraphics{Lista-2ED_files/figure-latex/składowe-główne-1} \end{center}

\begingroup\fontsize{10}{12}\selectfont

\begin{longtable}[t]{lr}
\caption{\label{tab:wykresy-obciazen}Największe obciążenia zmiennych na PC1 \label{tab:tabela2}}\\
\toprule
Zmienna & Loading\\
\midrule
Education & -0.403\\
Business.Freedom & -0.377\\
Environmental.Quality & -0.326\\
\bottomrule
\end{longtable}
\endgroup{}
\begingroup\fontsize{10}{12}\selectfont

\begin{longtable}[t]{lr}
\caption{\label{tab:wykresy-obciazen}Największe obciążenia zmiennych na PC2 \label{tab:tabela3}}\\
\toprule
Zmienna & Loading\\
\midrule
Startups & -0.483\\
Venture.Capital & -0.427\\
Leisure...Culture & -0.365\\
\bottomrule
\end{longtable}
\endgroup{}
\begingroup\fontsize{10}{12}\selectfont

\begin{longtable}[t]{lr}
\caption{\label{tab:wykresy-obciazen}Największe obciążenia zmiennych na PC3 \label{tab:tabela4}}\\
\toprule
Zmienna & Loading\\
\midrule
Commute & -0.506\\
Travel.Connectivity & -0.340\\
Safety & -0.333\\
\bottomrule
\end{longtable}
\endgroup{}

\textbf{PC1: Ogólny Poziom Rozwoju Społeczno-Gospodarczego}

Ta główna składowa łączy ze sobą edukację, wolność gospodarczą i jakość
środowiska. Ujemne obciążenia wskazują, że jednostki z wyższymi
wartościami PC1 charakteryzują się niższymi wynikami w tych trzech
obszarach. Możemy interpretować to tak, że PC1 odzwierciedla pewien
``potencjał'' lub ``bazowy poziom'' rozwoju. Wyższy wynik na tej
składowej może wskazywać na społeczeństwa, które mają solidne fundamenty
w edukacji, otwartą gospodarkę i dbałość o środowisko, co długoterminowo
sprzyja dobrobytowi. Warto zauważyć, że te trzy zmienne wzajemnie się
wzmacniają: wykształcone społeczeństwo może dążyć do lepszej jakości
środowiska i tworzyć bardziej innowacyjną i wolną gospodarkę.

\textbf{PC2: Innowacyjność i kultura}

Ta składowa grupuje startupy, kapitał wysokiego ryzyka oraz aktywności
związane z czasem wolnym i kulturą. Ujemne obciążenia sugerują, że
wyższe wartości tej składowej są powiązane z mniejszą liczbą startupów,
ograniczonym dostępem do finansowania innowacji (kapitał wysokiego
ryzyka) oraz uboższą ofertą kulturalną. Zatem, PC2 wydaje się mierzyć
``dynamikę'' i ``nowoczesność'' ekosystemu. Regiony z wyższymi wynikami
na tej składowej prawdopodobnie posiadają silne środowisko wspierające
nowe przedsięwzięcia, dostęp do inwestycji i bogate życie kulturalne, co
przyciąga talent i sprzyja innowacjom.

\textbf{PC3: Infrastruktura, logistyka, bezpieczeństwo}

Ta składowa łączy czas dojazdu, łączność komunikacyjną i poczucie
bezpieczeństwa. Ponownie, ujemne obciążenia oznaczają, że wyższe
wartości PC3 korelują z dłuższymi czasami dojazdu, słabszą
infrastrukturą komunikacyjną i niższym poziomem bezpieczeństwa. PC3
można więc interpretować jako miarę ``sprawności'' i ``jakości życia
codziennego''. Wysokie wyniki na tej składowej wskazują na obszary,
gdzie infrastruktura transportowa jest dobrze rozwinięta, komunikacja
jest efektywna, a mieszkańcy czują się bezpiecznie, co znacząco wpływa
na komfort życia i potencjał gospodarczy.

\begin{longtable}[t]{lr}
\caption{\label{tab:zmiennosc-poszczegolnych-skladowych}Liczba składowych potrzebnych do wyjaśnienia 80% i 90% zmienności \label{tab:potrzebneSkladowe}}\\
\toprule
Poziom.wyjaśnionej.wariancji & Liczba.potrzebnych.składowych\\
\midrule
80\% & 7\\
90\% & 10\\
\bottomrule
\end{longtable}

Jak widać w tabeli

\begin{longtable}[t]{lrr}
\caption{\label{tab:tabelki-dla-skumulowanegoudzialu}Udział wariancji wyjaśnionej przez składniki główne}\\
\toprule
Składowa & Udział..proc. & Skumulowany.udział..proc.\\
\midrule
PC1 & 29.80 & 29.80\\
PC2 & 15.16 & 44.96\\
PC3 & 12.25 & 57.21\\
PC4 & 7.65 & 64.86\\
PC5 & 7.05 & 71.90\\
\addlinespace
PC6 & 5.65 & 77.55\\
PC7 & 4.06 & 81.62\\
PC8 & 3.90 & 85.52\\
PC9 & 3.43 & 88.95\\
PC10 & 2.50 & 91.45\\
\addlinespace
PC11 & 1.90 & 93.35\\
PC12 & 1.71 & 95.06\\
PC13 & 1.62 & 96.68\\
PC14 & 1.11 & 97.79\\
PC15 & 0.91 & 98.69\\
\addlinespace
PC16 & 0.73 & 99.42\\
PC17 & 0.58 & 100.00\\
\bottomrule
\end{longtable}

\end{document}
